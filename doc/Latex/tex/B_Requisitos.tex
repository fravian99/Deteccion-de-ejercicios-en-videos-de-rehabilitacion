\apendice{Especificación de Requisitos}

\section{Introducción}

\section{Objetivos generales}

\section{Catálogo de requisitos}
\subsection{Requisitos funcionales}
\begin{itemize}
		\item \textbf{RF-1} Gestión de ejercicios
	\begin{itemize}
		\item \textbf{RF-1.1}  El terapeuta podrá crear un ejercicio nuevo.
		\item \textbf{RF-1.2} El terapeuta podrá eliminar un ejercicio existente.
	\end{itemize}
	\item \textbf{RF-2} Selección de ejercicios
	\begin{itemize}
		\item \textbf{RF-2.1}  El terapeuta podrá ver los ejercicios que hay disponibles.
		\item \textbf{RF-2.2} El usuario podrá ver los ejercicios que hay disponibles.
		\item \textbf{RF-2.3} El usuario podrá seleccionar el ejercicio que se va a comparar.
	\end{itemize}
	\item \textbf{RF-3} Gestión de ficheros
	\begin{itemize}
		\item \textbf{RF-3.1}  El terapeuta podrá guardar los ficheros con los datos en la aplicación.
		\item \textbf{RF-3.2} El usuario podrá seleccionar el fichero de datos que se va a comparar.
	\end{itemize}
	\item \textbf{RF-4} Comparación de ejercicios 
	\begin{itemize} 
		\item \textbf{RF-4.1}  El usuario podrá ver una puntuación numérica del ejercicio.
	\end{itemize}
\end{itemize}
\section{Especificación de requisitos}


\begin{table}[p]
	\centering
	\begin{tabularx}{\linewidth}{ p{0.21\columnwidth} p{0.71\columnwidth} }
		\toprule
		\textbf{CU-1}    & \textbf{Creación de un nuevo ejercicio}\\
		\toprule
		\textbf{Versión}              & 1.0    \\
		\textbf{Autor}                & Alumno \\
		\textbf{Requisitos asociados} & RF-1.1 RF-2.1 \\
		\textbf{Descripción}          & Permite al terapeuta crear un ejercicio   \\
		\textbf{Precondición}         & Precondiciones (podría haber más de una) \\
		\textbf{Acciones}             &
		\begin{enumerate}
			\def\labelenumi{\arabic{enumi}.}
			\tightlist
			\item El terapeuta introduce el nombre del ejercicio.
			\item El terapeuta añade un fichero de datos.
			\item El terapeuta presiona el botón de confirmar.
		\end{enumerate}\\
		\textbf{Postcondición}        & Ninguna \\
		\textbf{Excepciones}          & Ninguna \\
		\textbf{Importancia}          & Alta \\
		\bottomrule
	\end{tabularx}
	\caption{CU-1 Creación de un nuevo ejercicio.}
\end{table}

\begin{table}[p]
	\centering
	\begin{tabularx}{\linewidth}{ p{0.21\columnwidth} p{0.71\columnwidth} }
		\toprule
		\textbf{CU-2}    & \textbf{Eliminación de un ejercicio}\\
		\toprule
		\textbf{Versión}              & 1.0    \\
		\textbf{Autor}                & Alumno \\
		\textbf{Requisitos asociados} & RF-1.2 RF-2.1 \\
		\textbf{Descripción}          & Permite al terapeuta eliminar un ejercicio   \\
		\textbf{Precondición}         & El terapeuta debe haber creado ejercicios previamente \\
		\textbf{Acciones}             &
		\begin{enumerate}
			\def\labelenumi{\arabic{enumi}.}
			\tightlist
			\item El terapeuta selecciona el ejercicio.
			\item El terapeuta presiona el botón de confirmar.
		\end{enumerate}\\
		\textbf{Postcondición}        & Ninguna \\
		\textbf{Excepciones}          & Ninguna \\
		\textbf{Importancia}          & Alta \\
		\bottomrule
	\end{tabularx}
	\caption{CU-2 Eliminación de un ejercicio.}
\end{table}

\begin{table}[p]
	\centering
	\begin{tabularx}{\linewidth}{ p{0.21\columnwidth} p{0.71\columnwidth} }
		\toprule
		\textbf{CU-3}    & \textbf{Selección de ejercicio}\\
		\toprule
		\textbf{Versión}              & 1.0    \\
		\textbf{Autor}                & Alumno \\
		\textbf{Requisitos asociados} & RF-2.2 RF-2.3 \\
		\textbf{Descripción}          & Permite al usuario ver los ejercicios disponibles y seleccionar el ejercicio que se va a comparar\\
		\textbf{Precondición}         & El terapeuta debe haber creado ejercicios previamente \\
		\textbf{Acciones}             &
		\begin{enumerate}
			\def\labelenumi{\arabic{enumi}.}
			\tightlist
			\item El usuario selecciona un ejercicio de un listado.
		\end{enumerate}\\
		\textbf{Postcondición}        & Ninguna \\
		\textbf{Excepciones}          & Ninguna \\
		\textbf{Importancia}          & Alta \\
		\bottomrule
	\end{tabularx}
	\caption{CU-3 Selección de un nuevo ejercicio.}
\end{table}

\begin{table}[p]
	\centering
	\begin{tabularx}{\linewidth}{ p{0.21\columnwidth} p{0.71\columnwidth} }
		\toprule
		\textbf{CU-4}    & \textbf{Comparación de ejercicio}\\
		\toprule
		\textbf{Versión}              & 1.0    \\
		\textbf{Autor}                & Alumno \\
		\textbf{Requisitos asociados} & RF-4.1 \\
		\textbf{Descripción}          & Permite al usuario comparar el ejercicio\\
		\textbf{Precondición}     &    
		\begin{enumerate}		
			\def\labelenumi{\arabic{enumi}.}
			\tightlist
			\item El usuario ha seleccionado un ejercicio.
			\item El usuario ha seleccionado un fichero.
			\item El fichero se ha cargado correctamente.
		\end{enumerate}\\
		\textbf{Acciones}             &
		\begin{enumerate}
			\def\labelenumi{\arabic{enumi}.}
			\tightlist
			\item El usuario visualiza la puntuación de un ejercicio.
			\item El usuario visualiza el video del ejercicio.
		\end{enumerate}\\
		\textbf{Postcondición}        & Ninguna \\
		\textbf{Excepciones}          & Ninguna \\
		\textbf{Importancia}          & Alta \\
		\bottomrule
	\end{tabularx}
	\caption{CU-4 Comparación de ejercicio.}
\end{table}

\begin{table}[p]
	\centering
	\begin{tabularx}{\linewidth}{ p{0.21\columnwidth} p{0.71\columnwidth} }
		\toprule
		\textbf{CU-5}    & \textbf{Cargar fichero de ejercicio}\\
		\toprule
		\textbf{Versión}              & 1.0    \\
		\textbf{Autor}                & Alumno \\
		\textbf{Requisitos asociados} & RF-3.1\\
		\textbf{Descripción}          & Permite al usuario ver los ejercicios disponibles y seleccionar el ejercicio que se va a comparar\\
		\textbf{Precondición}         & El usuario a seleccionado un ejercicio \\
		\textbf{Acciones}             &
		\begin{enumerate}
			\def\labelenumi{\arabic{enumi}.}
			\tightlist
			\item El usuario selecciona un fichero.
			\item El usuario presiona el botón de confirmar.
		\end{enumerate}\\
		\textbf{Postcondición}        & El usuario ha seleccionado un fichero correctamente. \\
		\textbf{Excepciones}          & Ninguna \\
		\textbf{Importancia}          & Alta \\
		\bottomrule
	\end{tabularx}
	\caption{CU-5 Cargar fichero de ejercicio}
\end{table}

