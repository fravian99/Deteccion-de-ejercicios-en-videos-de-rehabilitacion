\apendice{Plan de Proyecto Software}

\section{Introducción}

\section{Planificación temporal}

\subsection{Sprint 0: 06/02/2024 - 12/03/2024}
En el \textit{sprint 0} se buscaron y leyeron artículos mientras se preparaban los vídeos necesarios. También se creo el repositorio y se clonó la plantilla de \LaTeX proporcionada por la Universidad De Burgos.

\begin{table}[H]
	\begin{tabular}{lcc}
		\multicolumn{1}{c}{Tareas} & Pto. Estimados & Pto. Reales\\
		Investigar sobre el sistema de puntuaciones de Just Dance & 8 & 9 \\
	 	Investigar sobre métodos para diferenciar las poses. & 5 & 8 \\
	 	Descargar plantilla de \LaTeX & 0.5 & 0.5 \\
	\end{tabular}
\caption{Sprint 0}
\label{sprint0}
\end{table}

\subsection{Sprint 1: 15/03/2024 - 26/03/2024}
En este \textit{sprint 1} se usó para hacer pruebas con los datos.

\begin{table}[H]
	\begin{tabular}{lcc}
		\multicolumn{1}{c}{Tareas} & Pto. Estimados & Pto. Reales\\
		Añadir archivos de datos & 0.5 & 0.5 \\
		Revisar vídeos de los ejercicios & 1 & 2 \\
		Dibujar esqueleto con los puntos de un frame & 3 & 4 \\
		Rotar esqueleto & 1 & 1 \\
		Normalización L2 & 2 & 1 \\
		Aplicar similitud coseno & 2 & 2 \\
		
	\end{tabular}
	\caption{Sprint 1}
	\label{sprint1}
\end{table}

\subsection{Sprint 2: 26/03/2024 - 11/04/2024}
\begin{table}[H]
	\begin{tabular}{lcc}
		\multicolumn{1}{c}{Tareas} & Pto. Estimados & Pto. Reales\\
		Explicar similitud coseno en la memoria & 3 & 3 \\
		Añadir Sprint 0 a la memoria & 1 & 1 \\
		Añadir Sprint 1 a la memoria & 1 & 1 \\
		Añadir herramientas a la memoria & 2 & 2 \\
		Mirar DTW & 5 & 5 \\
		Limpiar los datos de los csv y dataframe & 1 & 0.5 \\

	\end{tabular}
	\caption{Sprint 2}
	\label{sprint2}
\end{table}

\subsection{Sprint 3: 11/04/2024 - 24/04/2024}
\begin{table}[H]
	\begin{tabular}{lcc}
		\multicolumn{1}{c}{Tareas} & Pto. Estimados & Pto. Reales\\
		Comparación de videos mediante similitud coseno & 5 & 5 \\
		Pruebas con DTW & 5 & 5 \\
		Añadir Sprint 2 a la memoria & 1 & 1 \\
		Añadir explicación de DTW a la memoria & 5 & 5 \\
	\end{tabular}
	\caption{Sprint 3}
	\label{sprint3}
\end{table}

\subsection{Sprint 4: 26/04/2024 - 7/05/2024}
\begin{table}[H]
	\begin{tabular}{lcc}
		\multicolumn{1}{c}{Tareas} & Pto. Estimados & Pto. Reales\\
		Comprobar en que ficheros hay celdas vacías & 3 & 3 \\
		Comprobar 3 videos de un mismo ejercicio & 5 & 4 \\
		Añadir Sprint 3 a la memoria & 1 & 0.5 \\
	    DTW con ángulos& 5 & 6 \\
	\end{tabular}
	\caption{Sprint 4}
	\label{sprint4}
\end{table}

\subsection{Sprint 5: 8/05/2024 - 21/06/2024 }
\begin{table}[H]
	\begin{tabular}{lcc}
		\multicolumn{1}{c}{Tareas} & Pto. Estimados & Pto. Reales\\
	    Añadir datos de los ejercicios realizados correctamente y hacer pruebas con ellos & 2 & 2 \\
		Añadir pruebas con métodos de normalización en dtw & 3 & 5 \\
		Añadir Sprint 4 a la memoria & 1 &1 \\
		Reorganizando y añadiendo explicaciones en las pruebas & 3 & 3 \\
		Pruebas fastdtw & 3 & 3 \\
	\end{tabular}
	\caption{Sprint 5}
	\label{sprint5}
\end{table}

\subsection{Sprint 6: 22/05/2024 - 5/06/2024 }
\begin{table}[H]
	\begin{tabular}{lcc}
		\multicolumn{1}{c}{Tareas} & Pto. Estimados & Pto. Reales\\
		Investigar Angular, Fastapi y otras tecnologías para la aplicación & 8 & 8 \\
		Investigar métodos de normalización & 5 & 5 \\
		Añadir Sprint 5 a la memoria & 0.5 &0.5 \\
		Continuar con las pruebas de ángulos & 3 & 3 \\
	\end{tabular}
	\caption{Sprint 6}
	\label{sprint6}
\end{table}

\section{Estudio de viabilidad}

\subsection{Viabilidad económica}

\subsection{Viabilidad legal}


