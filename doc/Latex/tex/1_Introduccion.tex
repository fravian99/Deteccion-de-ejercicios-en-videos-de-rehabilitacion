\capitulo{1}{Introducción}

La enfermedad de Parkinson es una enfermedad neurodegenerativa crónica que no tiene cura actualmente. Esta enfermedad causa síntomas motores como pueden ser la realización de movimientos involuntarios, temblores o la bradicinesia\footnote{La bradicinesia consiste en la ralentización del movimiento o velocidad \cite{bradicinesia}.}, pero también provoca síntomas no motores como son el deterioro cognitivo, los trastornos mentales y la demencia.  

La enfermedad tiene un riesgo de incidencia distinto en función del sexo y la edad, afectando en el caso del primero más a los hombres que a las mujeres. En el caso de la edad, la mayoría de personas a las que afecta son de más de 60 años, mientras que entre un 5 y un 10\% de los casos se inicia antes de los 50 años \cite{parkinson:causas}.

Esta enfermedad, según estimaciones de 2019 de la Organización Mundial de la Salud afecto a 8,5 millones de personas y causo 329000 muertes desde 2000 \cite{parkinson:oms}.

Uno de los tratamientos para esta enfermedad es la fisioterapia, que ayuda a retrasar el avance de la enfermedad, para ello, se realizan ejercicios de rehabilitación moviendo distintas partes del cuerpo como pueden ser articulaciones. Estos ejercicios son evaluados por un profesional que busca fallos en la ejecución de dichos ejercicios. 

Estas sesiones típicamente son presenciales, lo que puede ser un problema para los pacientes, que suelen ser personas mayores y con problemas de movilidad, por lo que es necesario que sus familias se adapten a trasladar al paciente al centro de salud en el que vaya a realizar los ejercicios, que en el caso de las zonas rurales puede estar a varios kilómetros, haciendo imposible el traslado andando. Además, dependen de la disponibilidad de un terapeuta. Esto puede ser un problema debido a la falta de personal sanitario, aun más en el caso de las zonas rurales. Ademas, con el aumento de la esperanza de vida, cada vez habrá más personas con Parkinson, ya que afecta más a las personas mayores, por lo que puede agravarse más la situación \cite{parkinson:impact}.  

Una solución a esto es la telerehabilitación, que consiste en que el paciente se graba haciendo los ejercicios y se realiza un análisis sobre ellos. Esto hace que ya no sea necesario que los pacientes se tengan que desplazar ni que dependan de nadie para que les lleve, simplemente tendrían que conectarse a la app.

Este trabajo pretende encontrar una forma de automatizar dicho proceso dando una valoración al usuario de cuan bien se ha realizado el ejercicio comparando ese vídeo con el de un profesional.

Por ultimo, para poder ver los resultados, se desarrollará una aplicación web para que el usuario pueda ver la puntuación de sus ejercicios. 

Este trabajo hubiera sido imposible de realizar sin los resultados que se obtuvieron de los trabajos de fin de máster de José Miguel Ramírez Sanz y José Luis Garrido Labrador, tutores de este trabajo y del trabajo de fin de grado de Lucía Nuñéz Calvo, gracias a ellos se pudo contar con ficheros de datos con las posiciones extraídas de los vídeos recortados por ejercicio y de los ángulos calculados a partir de dichas posiciones.

\subsection{Estructura de la memoria}
La presente memoria se compone de las siguiente secciones:
\begin{enumerate}
	\item \textbf{Introducción:} Descripción del Parkinson y breve introducción al trabajo.
	\item \textbf{Objetivos del proyecto:} Se muestran los objetivos funcionales, técnicos y personales que motivan el desarrollo del proyecto.
	
	\item \textbf{Conceptos teóricos:} Se muestran definiciones y explicaciones de los distintos conocimientos necesarios para el entendimiento de este trabajo.
	
	\item \textbf{Técnicas y herramientas:} Conjunto de herramientas y metodologías elegidas para el desarrollo del proyecto.
	
	\item \textbf{Aspectos relevantes del desarrollo:} Explicación de las etapas más importantes del proyecto y problemas que surgieron durante el avance del mismo.
	
	\item \textbf{Trabajos relacionados:} Muestra trabajos y proyectos similares a este.
	
	\item  \textbf{Conclusiones y Líneas de trabajo futuras:} Conclusiones extraídas durante el proyecto y posibles áreas de mejora.	
\end{enumerate}

\subsection{Estructura de los anexos}
\begin{enumerate}
	\item  \textbf{Plan de proyecto:}  Descripción de los distintos sprints y estudio de viabilidades económicas y legales del proyecto.
	\item  \textbf{Especificación de los requisitos:} Requisitos y casos de uso del proyecto.
	\item  \textbf{Especificación de diseño:} Conjunto de diseños realizados para la correcta implementación del código.
	\item  \textbf{Documentación técnica de programación:} Descripción del repositorio, así como las fases de instalación y ejecución del proyecto.
	\item  \textbf{Documentación de usuario:}  Los manuales de usuario de la aplicación desarrollada.
\end{enumerate}