\capitulo{3}{Conceptos teóricos}


\section{Normalización} 
La normalización en el contexto de la ciencia de datos es el proceso para escalar los datos y transformarlos de forma que esten en un rango comun. 
\subsection{Normalización Min-Max}
La normalización mini-max es un método que permite estandarizar valores de forma que estén entre el 0 y el 1. 
\begin{equation}
	N_i = \frac{X_i - X_{min}}{X_{max} - X_{min}}
\end{equation}
\subsection{Normalización Z-score}
\begin{equation}
	N_i = \frac{X_i - \mu}{\sigma}
\end{equation}

\subsection{Normalización L2}
La normalización L2 es un método que se utiliza para transformar los puntos en un vector unitario. La suma de los cuadrados de las posiciones normalizadas sera igual a 1.

\section{Similitud coseno}
La similitud coseno permite diferenciar dos vectores, se utiliza para búsqueda y recuperación de información y comparación de documentos.
Es uno de los métodos para diferenciar dos posturas de dos imágenes concretas por medio de la similitud de los vectores de las distintas partes del cuerpo.

\begin{equation}
	\cos \theta = \frac{\vec{a} \cdot \vec{b}}{\lVert \vec{a} \lVert \cdot \lVert \vec{b} \lVert}
\end{equation}

Si el valor de este fuera 1 es que la imagen sería igual, mientras que si fuese 0 serían ortogonales, es decir que no comparten ninguna similitud.

\section{Dynamic time warping}
La deformación dinámica permite comparar dos secuencias temporales en la que se realizan los movimientos a distintas velocidades.

\imagen{img/warp}{Alineamiento de dos secuencias mediante DTW}{0.8}

\subsection{Matriz de costes locales}
Esta técnica se realiza mediante la comparación de las distancias de todos los pares de puntos en dos secuencias. Una distancia menor implica que estos puntos pueden ser candidatos a ser emparejados. \cite{dwt:dwtdescription}


\subsection{Matriz de costes acumulados}
Una vez emparejados los puntos se usa una matriz de costes acumulados.\cite{s19132882}

En la matriz de costes acumulados se inicializan los valores de la siguiente manera:
\begin{enumerate}
	\item  La primera fila:
	\begin{equation}
		D(1,j) = C(1,j)
	\end{equation}
	\item La primera columna:
	\begin{equation}
		D(i,1) = \sum_{k=1}^{i}C(k,1)
	\end{equation} 
	\item El resto:
	\begin{equation}
		D(i,j) = \min\{D(i-1, j-1),D(i-1,j),D(i,j-1)\} + C(i,j)
	\end{equation}
\end{enumerate}