\capitulo{2}{Objetivos del proyecto}

\section{Objetivos generales}
\begin{itemize}
	\item Investigar sobre la comparación de secuencias temporales.
	\item  Investigar distintos métodos de normalización para poder preparar los datos.
	\item Investigar sobre las distintas formas de evaluar la ejecución de ejercicios.
	\item Desarrollar una aplicación con la que el usuario pueda comparar ejercicios de forma sencilla.
\end{itemize}
\section{Objetivos técnicos}
\begin{itemize}
	\item Desarrollar un algoritmo en \textit{Python} que permita comparar dos ejercicios.
	\item Utilizar \LaTeX y \textit{TexStudio} para escribir la memoria.
	\item Utilizar un sistema de control de versiones como es \textit{Git} utilizando \textit{Github} para guardar el proyecto en remoto.
	\item Utilizar \textit{Jupyter Notebook} para mostrar los resultados de la investigación.
	\item Utilizar metodologías ágiles y marcos de trabajo durante el desarrollo como \textit{Scrum}. 
	
\end{itemize}
\section{Objetivos personales}
\begin{itemize}
	\item Realizar un aporte que pueda ayudar a personas con \textit{Parkinson}.
	\item Poner en práctica varios de los conocimientos adquiridos durante la carrera.
	 \item Introducirme en el desarrollo web aprendiendo a crear interfaces gráficas agradables para el usuario.
	\item Aprender sobre \textit{Docker} y \textit{Nginx}.
	\item Aprender tecnologías relacionadas con el desarrollo del \textit{frontend} aplicaciones web como son los lenguajes \textit{Typescript, HTML y CSS} y el \textit{framework} Angular. 
	\item Aprender tecnologías relacionadas con el desarrollo del \textit{backend} aplicaciones web como son \textit{FastAPI y SQLAlchemy}.
\end{itemize}