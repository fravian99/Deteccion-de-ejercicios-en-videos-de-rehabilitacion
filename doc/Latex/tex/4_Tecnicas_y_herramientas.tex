\capitulo{4}{Técnicas y herramientas}

En este apartado se explicarán las herramientas utilizadas para el trabajo.

\section{\LaTeX}
\LaTeX es un sistema de composición de textos orientado a la creación de documentos escritos que presenten una alta calidad tipográfica. ~\cite{wiki:latex}

\section{TeXstudio}
\textit{TeXstudio} es un editor de \LaTeX de código abierto y multiplataforma. ~\cite{wiki:textudio} 
Ofrece la posibilidad de escribir la memoria de forma local fácilmente. También se contempló la alternativa de usar Overleaf, ya que no requiere instalación, pero finalmente se eligió por TeXstudio ya que al simplificaba el no tener que descargar el documento cada vez que se quisiera actualizar la versión en el repositorio de Github.

\section{Git}
\textit{Git} es un programa de control de versiones. 
Muy útil para poder ver los cambios que se van realizando a medida que avanza el proyecto así como volver a versiones anteriores para deshacer cambios.

\section{Github}
\textit{Github} es una plataforma online que utiliza Git para guardar repositorios y ajustar la visibilidad de los mismos. 

\section{Gittyup}
\textit{Gittyup} permite utilizar Git a través de una interfaz gráfica, es una alternativa a programas como Fork que no están disponibles en Linux. Además, es de código abierto.

\section{Zube}
\textit{Zube} es una página web dirigida a la planificación de proyectos utilizando metodologías ágiles como \textit{Scrum} y \textit{Kanban}.

\section{Visual Studio Code}
\textit{Visual Studio Code} (también llamado VS Code) es un editor de código fuente desarrollado por Microsoft para Windows, Linux, macOS y Web. \cite{wiki:visualstudiocode}

\section{Detectron2}
\textit{Detectron2} es una librería de python destinada a la detección y segmentación de imágenes. 